\documentclass[11pt,a4paper]{article}

\usepackage{german,anysize,amsmath,amssymb,amsthm,paralist, array}
\usepackage{url}
\selectlanguage{german}

\pagestyle{empty}

\setlength{\textheight}{27cm}
\setlength{\parindent}{0cm}




\begin{document}
\textsc{Technische Universit"at Berlin}{\small\hfill 
AIM3: Scalable Data Mining and Data Analysis}\\
{\small Database Systems and Information Management Group{\small\hfill Summer term 2012}\\
Prof.~Dr.~Volker~Markl, Sebastian Schelter, Christoph Boden, Thomas Bodner}

\bigskip
\centerline{\Large\textbf{Third Assignment}}
\centerline{\emph{Linear Algebra on Parallel Processing Platforms}}
\centerline{Due on May 30th}
\bigskip

\centerline{\textbf{Search as Linear Algebra}}
\bigskip

\begin{enumerate}
\item \textbf{Search as matrix vector multiplication}

We have seen that search can be expressed as multiplication of the corpus matrix with a query vector. In \textit{de.tuberlin.dima.aim3.assignment3.SearchAsMatrixVectorMultiplication} you have to implement the vectorization of the corpus,
which is supplied in a textfile holding the following data per line:

\textit{documentID;terms}.

A query file is supplied that needs to be mapped to the vector for the multiplication. The result of the job is a vector 
whose entries denote the number of matched terms for each document.

\textit{Hint}: Use \textit{de.tuberlin.dima.aim3.assignment3.Dictionary} to map terms to matrix dimensions.

\item \textbf{Inverting an index as matrix transposition}

When searching documents, one usually builds a so called \textit{Inverted Index}, a data structure that for each term holds the documents containing it.
Conceptually the inverted index is equivalent to the transposed corpus matrix, which points from terms to documents. In \textit{de.tuberlin.dima.aim3.assignment3.MatrixTransposition} you have to implement matrix transposition for this task.

\end{enumerate}

\bigskip
\centerline{\textbf{Deadline}}
\bigskip

Source code for the exercises is available at \textit{https://github.com/dimalabs/scalable-datamining-class}. 
\\
\\
Upload your solution to ISIS in the form of a patch file until noon of May 30th.

\end{document}
